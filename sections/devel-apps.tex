\section*{\textcolor{maroon}{\normalsize Developed Systems and Applications}} 

\begin{enumerate}

\item \textbf{\href{https://www.eleicoessemfake.dcc.ufmg.br}{Eleições sem Fake:}} We developed many systems to help with the Fake news problem. Those systems were the key to our research on misinformation dissemination within public Whatsapp groups [WWW'19], [ICWSM'19], [WebMedia'18].
\item \textbf{\href{https://twitter-app.mpi-sws.org/who-makes-trends/}{Who Makes Trends?:}} Demographic of Trend Promoters is the distribution (or combination) of demographic groups (such as middle-aged white men, young asian women, adolescent black men) in the crowd promoting (or posting about) a topic before the topic becomes Trending on Twitter. Here, we are only considering US based Twitter users whose tweets on the trends appear in the $1\%$ random sample distributed by Twitter. This system was key to our work published on [ICWSM'17].
\item \textbf{\href{http://twitter-app.mpi-sws.org/search-political-bias-of-users/}{Search Political Leaning of Twitter Users:}} You can login with your Twitter credentials, to see the political leaning (between democratic and republican) inferred for you. You can also search for other Twitter users and check their political leanings. This system was key to our work published on [CSCW'17] and [Inf Retrieval J'19].
\item \textbf{\href{http://twitter-app.mpi-sws.org/footprint/}{Secondary Digital Footprint:}} Twitter is social, people converse with you by mentioning your username in their tweets (e.g., while replying to your tweet or giving a shout-out to you ). These conversations are your secondary digital footprint , even if you delete your account or delete selected tweets, this secondary footprint is not deleted automatically and leaks information about you. Check what your secondary digital footprint reveals about you and your content. This system was key to our research published on [SOUPS'16], [IEEE Internet Computing'17], and [Int J Adv Eng Sci Appl Math'17].

\end{enumerate}

%%% Local Variables:
%%% mode: latex
%%% TeX-master: "../paper"
%%% End:
