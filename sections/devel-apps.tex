\section*{\textcolor{maroon}{\normalsize Developed Systems and Applications}} 

\begin{enumerate}

\item \textbf{\href{https://www.eleicoessemfake.dcc.ufmg.br}{Eleições sem Fake:}} We developed multiple systems to help with the Fake News problem. Those systems were the key to our research on misinformation dissemination within public WhatsApp groups [WWW'19], [ICWSM'19], [WebMedia'18].
\item \textbf{\href{https://twitter-app.mpi-sws.org/who-makes-trends/}{Who Makes Trends?:}} Demographic of Trend Promoters is the distribution (or combination) of demographic groups (such as middle-aged white men, young asian women, adolescent black men) in the crowd promoting (or posting about) a topic before the topic becomes Trending on Twitter. This system was key to our work published in [ICWSM'17].
\item \textbf{\href{http://twitter-app.mpi-sws.org/search-political-bias-of-users/}{Search Political Leaning of Twitter Users:}} This system helps users to infer the political leaning (between democratic and republican) of Twitter. It was key to our work published in [CSCW'17] and [Inf Retrieval J'19].
\item \textbf{\href{http://twitter-app.mpi-sws.org/footprint/}{Secondary Digital Footprint:}} Twitter is social, people converse with other users by mentioning their username in their own tweets (e.g., while replying to other Twitter user's tweet or giving a shout-out to them). These conversations are their secondary digital footprint, even if they delete their account or delete selected tweets, this secondary footprint is not deleted automatically and leaks information about them. This system was key to our research published in [SOUPS'16], [IEEE Internet Computing'17], and [Int J Adv Eng Sci Appl Math'17].

\end{enumerate}

%%% Local Variables:
%%% mode: latex
%%% TeX-master: "../paper"
%%% End:
